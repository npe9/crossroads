%%
\RequirePackage{flashmovie}
\documentclass[t]{beamer}
\usepackage[english]{babel}           % or whatever
\usepackage[latin1]{inputenc}         % or whatever
\usepackage{times}
\usepackage[T1]{fontenc}
\usepackage{url}
\usepackage{multimedia}
\usepackage{hyperref}
\usepackage{verbatim}
\usepackage{graphicx,natbib,amssymb,amsmath}
\usepackage[absolute,overlay]{textpos}

\newcommand{\be}{\begin{equation}}
\newcommand{\ee}{\end{equation}}
\newcommand{\bes}{\begin{equation*}}
\newcommand{\ees}{\end{equation*}}
\newcommand{\iphj}{{i+\frac 1 2,j}}
\newcommand{\ipoj}{{i+1,j}}
\newcommand{\iphjpo}{{i+\frac 1 2,j+1}}
\newcommand{\iphjpt}{{i+\frac 1 2,j+\frac 3 2}}
\newcommand{\ijph}{{i,j+\frac 1 2}}
\newcommand{\ijpo}{{i,j+ 1}}
\newcommand{\ipojph}{{i+1,j+\frac 1 2}}
\newcommand{\iptjph}{{i+\frac 3 2 ,j+\frac 1 2}}
\newcommand{\iptjpo}{{i+\frac 3 2 ,j+1 }}
\newcommand{\ipojpt}{{i+1 ,j+\frac 3 2}}
\newcommand{\iptjpt}{{i+\frac 3 2 ,j+\frac 3 2}}
\newcommand{\imhj}{{i-\frac 1 2,j}}
\newcommand{\ijmh}{{i,j-\frac 1 2}}
\newcommand{\iphjph}{{i+\frac 1 2,j+\frac 1 2}}
\newcommand{\iphjmh}{{i+\frac 1 2,j-\frac 1 2}}
\newcommand{\imhjph}{{i-\frac 1 2,j+\frac 1 2}}
\newcommand{\nmh}{{n-\frac 1 2}}
\newcommand{\nph}{{n+\frac 1 2}}
\newcommand{\nmo}{{n-1}}
\newcommand{\npo}{{n+1}}
\newcommand{\npth}{{n+\frac 3 2}}
\newcommand{\dx}{{\triangle x}}
\newcommand{\dy}{{\triangle y}}
\newcommand{\dz}{{\triangle z}}
\newcommand{\dt}{{\triangle t}}
\newcommand{\partder}[2]{ \frac { \partial {#1} } {\partial {#2} }}



\mode<presentation>
{
  \usetheme{Sandia} % or ...
  \setbeamercovered{transparent} % or whatever (possibly just delete it)
  \usefonttheme[onlymath]{serif} % use normal math fonts
}

% This causes an updated outline to appear before each section
\mode<beamer>
{
  \AtBeginSection[]{
    \begin{frame}
      \frametitle{Outline}
       {\small\tableofcontents[currentsection,hideothersubsections]}
    \end{frame}
  }


}
\defbeamertemplate{itemize item}{myball}%
{\raise-0.2cm\beamer@usesphere{item projected}{mysphere}}

% PKN: the footline isn't very useful -- better to conserve pixels
%\setbeamertemplate{footline}{}
\setbeamertemplate{navigation symbols}{}

\title{{mini-PIC}, AN UNSTRUCTURED FINITE ELEMENT PARTICLE-IN-CELL
  CODE FOR THREADED APPLICATIONS}

\author{Matthew T. Bettencourt}
\institute{Sandia National Laboratories}

\date{Today}

\subject{Important Material}

\begin{document}

%\setbeamertemplate{itemize item}{\small\raise1.25pt}

%%%%%%%%%%%%%%%%%%%%%%%%%%%%%%%%%%%%%%%%%%%%%%%%%%%%%%%%%%%%%%%%%%%%%%%%%%

\begin{frame}
  \titlepage
\end{frame}

%%%%%%%%%%%%%%%%%%%%%%%%%%%%%%%%%%%%%%%%%%%%%%%%%%%%%%%%%%%%%%%%%%%%%%%%%%

\begin{frame}
  \frametitle{Outline}
  {\small\tableofcontents[hideallsubsections]}
\end{frame}

%%%%%%%%%%%%%%%%%%%%%%%%%%%%%%%%%%%%%%%%%%%%%%%%%%%%%%%%%%%%%%%%%%%%%%%%%%

\section{What are Particle-in-Cell {(PIC)} codes}

\begin{frame}
 \frametitle{What Problems Do They Solve}
  \begin{itemize}
  \item PIC is used for when you represent your physics in two parts
  \begin{itemize}
    \item Discrete physical representation of something - a particle
    \item Background field
  \end{itemize}
  \item Examples are:
  \begin{itemize}
  \item Planets/stars are the ``particles' with a background field
    being gravity
  \item Center of a fluid vortex with the field being global flow field
  \item {\bf{Plasma dynamics with charged particles (electrons/ions)
    creating a global electric field}}
  \end{itemize}
  \item Solution technique
  \begin{itemize}
  \item Particles are represented as discrete Lagrangian entities
    which can migrate freely through the simulation domain
  \item Fields are represented on the mesh, typically these are
    computed by solving a potential type equation
  \begin{itemize}
    \item {\bf{For plasma, Poisson's equation for electric potential
      based on charge density} }
  \end{itemize}

  \item Particles and Fields are coupled 
  \item PIC codes often involve stochastic processes
  \end{itemize}
  \end{itemize}
\end{frame}

\begin{frame}
 
 \frametitle{mini-PIC Update Scheme}
  \begin{itemize}
  \item We are solving the electrostatic Boltzmann equations 
\bes
\partder{f_i}{t} +\vec{v} \cdot \partder{f_i}{\vec{x}} + \frac{\vec{F}}{m}
\cdot  \partder{f_i}{\vec{v}} = 0 ;
\hspace{.2in}
\nabla \cdot \vec E = \frac \rho \epsilon = \sum^N_{i=1} q \frac{f_i}{\epsilon}; 
\hspace{.2in}
F = q \vec E
\ees

  \item mini-PIC solves these equation on an unstructured
  grid in the  following steps
  \begin{enumerate}
  \item Particles are weighting to the grid as charge density $\rho$
  \item Poisson's equation is solved $\nabla ^2 \phi = \frac \rho \epsilon$
  \item Electric field is calculated $\vec E = -\nabla \cdot \phi$
  \item Field is weighted to the particle location
  \item Particle is accelerated $\vec F = q \vec E = m \vec a $
  \item Particle is moved to new location
  \item {\emph{Particles interact with walls or other particles (DSMC) - new
    particles are created}}
  \item Repeat
  \end{enumerate}
  \end{itemize}

\end{frame}


\begin{frame}
 
 \frametitle{What Makes PIC Different from FEM}
 \begin{itemize}
 \item PIC has two parts - Field solve just standard implicit FEM
 \item Particle part very different
 \begin{itemize}
 \item Particles are not fixed to an element, migrate through the system
 \begin{itemize}
 \item Particles need element data for forcing, charge deposition and
   element crossing
 \item Work can become unbalanced, particle mesh needs to be continually rebalanced
 \item Work (particles) are created dynamically, particle hits a wall, 3 more may come out 
 \item Gas chemistry - and electron hits a N atom, creates an $N^+$ and an $e^-$ based on a stochastic process.  
 \end{itemize}
 \item Many processes require pseudo random numbers
 \begin{itemize}
 \item Random number generators must be repeatable regardless of thread count
 \end{itemize}
 \end{itemize}
 \item Summarizing, dynamic memory requirements, localized work loads, stochastic processes
 \end{itemize}

\end{frame}

\section{What is mini-PIC}

\begin{frame}
 
 \frametitle{mini-PIC Overview}
 \begin{itemize}
 \item Solves the discrete Boltzmann equation in and
   electrostatic field in an arbitrary domain (pamgen mesh) with reflective walls
 \item Uses an unstructured hex mesh
\begin{itemize}
 \item Requires an iterative particle push based on skew of the element
 \end{itemize}

 \item Written with MPI parallelization at a high level, Kokkos at a node level
 \item Uses a static partition for a particle mesh
 \begin{itemize}
 \item Particles are tracked to every cell crossing, packed and passed off to adjacent processors when needed
 \end{itemize}
 \item Uses Tpetra for matrix/vector/map storage
 \item Has an unpreconditioned CG solver hand coded
 \item Kokkos only has been tested on GPU, CPU and MIC
 \item MPI+Kokkos only available on CPU/MIC - Tpetra doesn't work on GPU
 \end{itemize}
\end{frame}

\begin{frame}
 
 \frametitle{Particle Data Storage}
 \begin{itemize}
 \item Core data type is a Kokkos::Segmented\_Vector
 \begin{itemize}
 \item An array of chunks of data
 \item Allows for lock free dynamic addition and subtraction of chunks
 \item Requires a thread team to all add a chunk at the same time
  \item Only on the device, no deep\_copy or other way to bring data
    to the host - being resolved
 \end{itemize}
 \item ParticleList holds all the particle data
 \begin{itemize}
 \item Structure of arrays, position, velocity, force, ... (21 variables total)
 \item Has a team push\_back function, delete\_particles, and migrate\_particle functions
 \end{itemize}
 \end{itemize}
\end{frame}
\begin{frame}
 
 \frametitle{Particle migration and deletion}

 \begin{itemize}
   \item When a particle reaches the end of a parallel domain it's
     marked for deletion and the index of this particle is added to a
     migrate and deletion list
     \begin{itemize}
     \item A full thread team needs to all call this function as one
       might need to grow these lists
     \end{itemize}
   \item Once all the particles on a processor have moved their
     particles as far as they can
     \begin{itemize}
     \item All the particles from the migration list are packed into
       buffers and sent to their neighboring processors
     \item These particles are then deleted from the main list and the
       list is back-filled
     \item The new particles are unpacked and their temporal
       advancement is continued  
     \end{itemize}
     \item This is repeated until all the particles have been pushed
       to their final location
 \end{itemize}


\end{frame}


\section{Performance}


\begin{frame}
 
 \frametitle{Performance Tuning}

 \begin{itemize}
   \item {Mini-PIC has been ported to three platforms}
   \begin{itemize}
     \item {Pthreads on Intel Xeon, OpenMP on Intel Xeon Phi, and Cuda
     on Nvidia GPU}
   \end{itemize}
   \item {Kokkos made the porting simple, yet machine specific
       tuning was required}
  \begin{itemize}
     \item {Because particle creation and deletion requires a thread team to
       enter a particle update loop, this could lead to imbalance }
     \item {A custom dynamic scheduler was developed using atomics}
   \begin{itemize}
     \item {This scheduler was not used on Cuda platform because Cuda
       has efficient dynamic scheduling}
     \item {Kokkos is looking at having a dynamic scheduling policy}
   \end{itemize}
   \item {Charge weighting requires atomic operations}
   \begin{itemize}
     \item {Particles are typically clustered in memory by element,
       leads to atomic conflicts}
     \item {Randomization of particles helped reduce this }
     \item {Proper solution, sort particles by element, have a team
       handle an elements particles to local memory, then use atomics
       to push that to the mesh}
       \item {Can't use Kokkos::Sort with a segmented\_view}
   \end{itemize}

 
   \end{itemize}
  \end{itemize}
\end{frame}


\begin{frame}
 
 \frametitle{Performance Results}

 \begin{itemize}
   \item {mini-PIC tested on 2x6 Intel Xeon X5650, Xeon Phi KNC and Nvidia k20x}
     \item {Work breakdown roughly the same on }
 \begin{itemize}
   \item {60\% Particle Move}
   \item {\bf{30\% Particle Charge Weighting}}
   \item {10\% Electric Field Weighting}
   \item {$\approx$1\% Field Solve}
 \end{itemize}
 \item {Performance on the different machines is (higher is better) }
   \begin{itemize}
   \item {Intel Xeon {\bf{4.1M}} updates/second}
   \item {Intel Xeon Phi {\bf{4.8M}} updates/second}
   \item {Nvidia k20x {\bf{5.2M}} updates/second}
 \end{itemize}
 \item {Poor relative Cuda performance is due to atomic contention in charge
   weighting and imbalance in particle move due to high thread team
   count}
   \begin{itemize}
     \item Moving to tets from hexes would improve Cuda performance
       while hurting others
     \item Using a structured mesh would improve particle move performance by factor
       of 10
 \end{itemize}


 \end{itemize}


\end{frame}

\if 0

\section{Rolls for Task-Based Parallelism}
\begin{frame}
 \frametitle{Some Ideas for Task-Based Parallelism}

 \begin{itemize}
   \item The dynamic nature of PIC work leads to imbalance
   \item One can restructure the move to be better balanced
 \begin{itemize}
   \item Particles are tied to an element at the start of a time-step -
     Each element may have O(100) particles
   \item Restructure data structures so an element owns it's own
     particles
   \item Every element update is a task
   \item Particle migration to neighboring elements handled like off
     processor communication
   \item \bf{Advantage is that you would have better load-balancing,
       better data locality, more reuse}
   \item \bf{Disadvantage is that you would have to go through more
       iteration of an update as a particle can cross 10+ elements,
       more memory bloat, lots more complexity}

 \end{itemize}
 \end{itemize}


\end{frame}
\begin{frame}
 \frametitle{Some Ideas for Task-Based Parallelism}
 \begin{itemize}
 \item Thinking smaller
   \begin{itemize}
     \item Chunk up the particles into small tasks not tied to
       elements 
     \item Interleave move and migrate tasks
     \item When a migrate buffer is full, add a task to migrate that
       data to where it can be processed
       \item \bf{Simpler to implement and overlapped communication and
           computation}
       \item \bf{Still need to do larger rebalance step, worse data
           reuse, ...}

   \end{itemize}
     
   \item Thinking more global
   \begin{itemize}
      \item Different parts of the calculation are tasks
        \item Compute force, accelerate particle, move particle,
          weight particle to a grid.  These can be done completely for
          a group of particles as several distinct tasks
        \item This can be quite complicated to track where different
          particles are in updates as they are migrated across
          processors


 \end{itemize}
 \item \bf{All of these algorithms require migration of mesh data
     structures along with the particles so we can perform movements}
 \item \bf{An ideal decomposition for motion may not be any good for
     weighting to mesh elements for FEM solve}
 \end{itemize}


\end{frame}

\fi

\section{Summary}
\begin{frame}
\frametitle{Summary}
\begin{itemize}
\item mini-PIC is a small (~2500 lines) hybrid MPI/Kokkos application 
\item Performance near identical on different platforms tested, about
  5M updates/second
\item It has a dynamic work load making static partitioning and
  scheduling inefficient
\item Available today in a basic form
\begin{itemize}
\item Released under BSD license
\item Developing large scale test problem with analytic solution for
  pushing to http://mantevo.org
\end{itemize}
\end{itemize}
\end{frame}
%%%%%%%%%%%%%%%%%%%%%%%%%%%%%%%%%%%%%%%%%%%%%%%%%%%%%%%%%%%%%%%%%%%%%%%%%%

\end{document}
